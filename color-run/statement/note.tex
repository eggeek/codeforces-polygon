The first example can be interpreted as follow:
\begin{itemize}
  \item The first line - \t{n}: there are \t{n} cows.
  \item The next line - \t{3 1 2 3}: the first cow has \t{3} favorite colors, they are \t{1 2 3}.
  \item The next line - \t{2 1 3}: the second cow has \t{2} favorite colors, they are \t{1 3}.
  \item The next line - \t{3 2 4 5}: the third cow has \t{3} favorite colors, they are \t{2 4 5}.
\end{itemize}

Farmer John can run the machine to douse color \t{1} on the first and second cow,
and douse color \t{2} on the third cow, and there is no way to color all cows to same color,
    so the minimum number of operation is \t{2}.

