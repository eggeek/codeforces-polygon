Let's start with the most classic introduction problem: \t{A+B}.
To finish the task, you need to read one line and parse it to two integers,
in \t{Python3}, you can do this in one line: \t{a, b = map(int, input().split()}.

A common mistake is writing code like this: \t{a, b = map(int, intput("give input:").split())}.
Such code will produce extra output \t{"give input:"}, and the judge program will regard this as the part of result,
so that you will get \t{Wrong Answer}.
