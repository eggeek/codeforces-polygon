On Unix computers, data is stored in directories. There is one root directory, and this might
have several directories contained inside of it, each with different names. These directories
might have even more directories contained inside of them, and so on.

A directory is uniquely identified by its name and its parent directory (the directory it is
directly contained in). This is usually encoded in a path, which consists of several parts each
preceded by a forward slash ('/'). The final part is the name of the directory, and everything
else gives the path of its parent directory. For example, consider the path:

\t{/home/anzac/round4}

This refers to the directory with name \t{"round4"} in the directory described by \t{"/home/anzac"},
which in turn refers to the directory with name \t{"anzac"} in the directory described by the path
\t{"/home"}. In this path, there is only one part, which means it refers to the directory with the
name \t{"home"} in the root directory.

To create a directory, you can use the \t{mkdir} command. You specify a path, and then mkdir will
create the directory described by that path, but only if the parent directory already exists.
For example, if you wanted to create the \t{"/home/anzac/round4"} and \t{"/home/anzac/finals"} directories
from scratch, you would need four commands:

\begin{itemize}
  \item \t{mkdir /home}
  \item \t{mkdir /home/anzac}
  \item \t{mkdir /home/anzac/round4}
  \item \t{mkdir /home/anzac/finals}
\end{itemize}

Given the full set of directories already existing on your computer, and a set of new
directories you want to create if they do not already exist, how many mkdir commands do you
need to use?

A path consists of one or more lower-case alpha-numeric strings (i.e., strings containing only
the symbols 'a'-'z' and '0'-'9'), each preceded by a single forward slash. These alpha-numeric
strings are never empty.
